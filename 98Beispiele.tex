\section{\LaTeX Beispiele}\label{sec:bsp}

\subsection{Aufzählung und Nummerierung}

Zu den wichtigsten Aufgaben der Medium Access Control gehören:
\begin{itemize}
  \item Kanal-Scanning
  \item PAN Association/Disassociation
  \item 128-Bit-AES-Verschlüsselung
\end{itemize}
\begin{enumerate}
  \item FFD führt ein aktives Kanal-Scanning durch
  \item FFD wählt einen freien Kanal und eine freie LoWPAN-ID
  \item Das FFD wird zum LoWPAN-Coordinator
\end{enumerate}

\subsection{Verweise}

Verweis auf Abschnitt \ref{sec:bsp} \\
Zitat der Quelle \cite{gk09}

\subsection{Akronyme}
Die \gls{bts} ist das passive Radio des GSM-Netzes \\
und wir vom \gls{bsc} gesteuert.

\subsection{Abbildung}

\begin{figure}[!ht]
  \centering
  \includegraphics[width=0.85\textwidth]{img/hm.jpg}
  \caption{Logo der Hochschule München}
  \label{fig:HS}
\end{figure}

\newpage

\subsection{Tabelle}

\begin{table}[!ht]
  \centering  
  \begin{tabular}{|p{3cm}||p{4cm}|p{4cm}|}
    \hline \textbf{Parameter} & \textbf{6LoWPAN (CC2530)} & \textbf{BLE (CC
2540)} \\ 
    \hline
    \hline Frequenzband & 2,4\,GHz ISM-Band & 2,4\,GHz ISM-Band \\
    \hline Kanalzugriff & CSMA/CA & AFHSS \\
    \hline Verbindungstyp & verbindungslos & verbindungsorientiert \\
    \hline max. Stromaufnahme & \vspace{1pt}29\,mA & \vspace{1pt}24\,mA \\ 
    \hline Energieverbrauch (Sendevorgang) & \vspace{1pt}407\,$\mu$Wh &
\vspace{1pt}277\,$\mu$Wh \\ 
    \hline Daten-übertragungsrate & \vspace{1pt}250\,kBit/s &
\vspace{1pt}1\,MBit/s \\
    \hline max. Anzahl Kommunikationsknoten & \vspace{1pt}$2^{64}-2$ &
\vspace{1pt}$2^{32}-2$ \\
    \hline Topologien & Stern; Peer-to-Peer & Stern \\
    \hline Reichweite & ca. 100\,m & ca. 70\,m \\
    \hline Implementierung & große Auswahl; IPv6 möglich & stark begrenzte
Auswahl; IPv6 umständlich \\
    \hline max. Payload pro Daten-Frame & \vspace{1pt}137\,Byte &
\vspace{1pt}39\,Byte \\
    \hline Stackgröße (Flash/RAM) & \vspace{0.5pt}32\,kByte/4\,kByte &
\vspace{0.5pt}120\,kByte/ - \\
    \hline Datensicherheit & 128-Bit-AES keine implementierten Funktionen &
128-Bit-AES Security Manager \\
    \hline
  \end{tabular}
  \caption{Parametervergleich 6LoWPAN vs. Bluetooth Low Energy}\label{tab:ver}
\end{table}

\subsection{Code-Listing}

\begin{lstlisting}[caption=Beispiel Speicherallozierung,label=list:mal]
 /* malloc1.c */
#include <stdio.h>
#include <stdlib.h>

int main(void) {
   int *p;

   p = malloc(sizeof(int));
   if(p != NULL) {
      *p=99;
      printf("Allokationerfolgreich ... \n");
   }
   else {
      printf("Kein virtueller RAM mehr verfügbar ...\n");
      return EXIT_FAILURE;
   }
   return EXIT_SUCCESS;
}
\end{lstlisting}
